%Notações de Função
\documentclass[a4paper, 12pt]{article}
\usepackage[top=2cm, bottom=2cm, left=2.5cm, right=2.5cm]{geometry}
\usepackage[utf8]{inputenc}
\usepackage{amsmath, amsfonts, amssymb} 
\DeclareMathOperator{\sen}{sen} %declara o operador matemático seno (para exibir em português)

\begin{document}

\begin{enumerate}
    \item Seja a função 
    $$f : \mathbb{R} \to \mathbb{R}$$ %\to representa uma seta
    $$x \mapsto \dfrac{1}{2}x^2 - 2x + 1$$. %\mapsto representa outra seta
    
    \item Seja a função
    $$f:\mathbb{R} \to \mathbb{R}$$
    $$f(x) = 
    \begin{cases} %usado quando uma variável dependente assume funções diferentes, dependendo do valor da variável independente
        
        x^2 - 1; \, \textrm{se } x \geq 1 \\ %essas barras duplas pulam linha
        x - 3; \, \textrm{se } -1 \leq x < 1 \\  
        2x + 1; \, \textrm{se } x < -1
        
    \end{cases}
    $$
  
    \item Seja a função
    $$f: \mathbb{R} \to \mathbb{R^*_+}$$
    $$f(x) = 
    \begin{cases}
        \log_2 x \textrm{, se } x \geq 3\\ %função logarítimica
        \ln x \textrm{, se } x < 3
    \end{cases}
    $$
    
    \item Seja a função
    $$f : \mathbb{R} \to \mathbb{R}$$
    $$f(x) = \cos x$$ %funções trigonométricas
    $$f(x) = \sin x$$
    $$f(x) = \sen x$$ %essa função só existe porque foi declarada no preâmbulo
    
    \item Seja a função
    $$f : \mathbb{R} \to \mathbb{R}$$
    $$f(x) = \sin (x - \frac{\pi}{2})$$ %colocando parênteses que não cobrem a altura da fração
    $$f(x) = \sin \left(x - \frac{\pi}{2} \right)$$ %colocando parênteses que cubra a altura da fração
    $$f(x) = \sin \left[x - \frac{\pi}{2} \right]$$ %colocando colchetes que cubra a altura da fração
    $$f(x) = \sin \left\{x - \frac{\pi}{2} \right\}$$ %colocando colchetes que cubra a altura da fraçã 
    
    
    \begin{enumerate}
        \item Esboce o gráfico da função
    \end{enumerate}
    
\end{enumerate}

\end{document}