%Inserir Tabela
\documentclass[a4paper, 12pt]{article}
\usepackage[top=2cm, bottom=2cm, left=2.5cm, right=2.5cm]{geometry}
\usepackage[utf8]{inputenc}
\usepackage{amsmath, amsfonts, amssymb} 
\usepackage{graphicx}
\usepackage{float} 
\usepackage[portuguese]{babel}

\begin{document}

\begin{center}
    \begin{tabular}{|c|p{10cm}|} %uma maneira de aumentar a borda do cabeçalho
            \hline
            \begin{tabular}{c}
                \includegraphics[scale=0.25]{imagens/logotipo.jpeg}
            \end{tabular}
            
            &
            
            \begin{tabular}{r}
                Canal LCMAQUINO \\
                http://www.lcmaquino.org/ \\
                Prof. Luiz C. M. de Aquino. \\
            \end{tabular}
            
        \\ \hline
        
        \end{tabular}
\end{center}

\begin{center}
    \begin{tabular}{|p{0.6\textwidth}|c|} %outra maneira de aumentar a borda do cabeçalho
            \hline
            \begin{tabular}{c}
                \includegraphics[scale=0.25]{imagens/logotipo.jpeg}
            \end{tabular}
            &
            \begin{tabular}{r}
                Canal LCMAQUINO \\
                http://www.lcmaquino.org/ \\
                Prof. Luiz C. M. de Aquino. \\
            \end{tabular}
            
        \\ \hline
        
        \end{tabular}
\end{center}

\begin{enumerate}
    \item A tabela abaixo representa as derivadas básicas.
    
    \begin{tabular}{cc} %comando para colocar tabelas no texto. a letra c significa centralizar. ela pode ser substituída por r (direita) ou l (esquerda)
    
    Função & Derivada\\
    $f(x) = x^n$     & $f'(x) = nx^{n-1}$ \\
    $f(x) = \log_a x$ & $f'(x) = \dfrac{1}{x \ln a}$
    \end{tabular}
    
    \item A tabela abaixo representa as derivadas básicas.

    \begin{tabular}{|c|c|} %as barras verticais colocadas no argumento da função colocam de fato barras verticais no documento do latex
    
    \hline %coloca linhas horizontais abaixo do texto
    
    Função & Derivada\\ \hline 
    $f(x) = x^n$     & $f'(x) = nx^{n-1}$ \\ \hline
    $f(x) = \log_a x$ & $f'(x) = \dfrac{1}{x \ln a}$ \\ \hline
    \end{tabular}
    
    \item A tabela abaixo representa as derivadas básicas.

    \begin{tabular}{c||c} 
    \hline 
    Função & Derivada\\ \hline \hline 
    $f(x) = x^n$     & $f'(x) = nx^{n-1}$ \\ \hline \hline
    $f(x) = \log_a x$ & $f'(x) = \dfrac{1}{x (\ln a)}$ \\ \hline
    \end{tabular}
    
    \item A tabela abaixo representa as derivadas básicas.

    
    \begin{center} %centralizar tabela
        
        \begin{tabular}{c||c} 
        \hline 
        Função & Derivada\\ \hline \hline 
        $f(x) = x^n$     & $f'(x) = nx^{n-1}$ \\ \hline \hline
        $f(x) = \log_a x$ & $f'(x) = \dfrac{1}{x \ln a}$ \\ \hline
        \end{tabular}
        
    \end{center}
    
    \item A Tabela \ref{minha-tabela} representa as derivadas básicas.
    
    \begin{table}[!htb] %organiza a tabela, colocando legenda, numeração, essas coisas. o que está escrito entre conchetes "htb". o ponto de exclamação ! no colchete meio que força a quebrar as regras de htb e tenta colocar o máximo possível no local desejado. 
    
    \centering
    
        \begin{tabular}{c||c} 
        \hline 
        Função & Derivada\\ \hline \hline 
        $f(x) = x^n$     & $f'(x) = nx^{n-1}$ \\ \hline \hline
        $f(x) = \log_a x$ & $f'(x) = \dfrac{1}{x (\ln a)}$ \\ \hline
        \end{tabular}
    
    \caption{Tabela básica de Derivadas.} %inserir legenda
    \label{minha-tabela} %colocar umm rótulo na tabela (para referenciar no texto).
    \end{table}
    
    
\end{enumerate}

\end{document}