%Notações de Matrix
\documentclass[a4paper, 12pt]{article}
\usepackage[top=2cm, bottom=2cm, left=2.5cm, right=2.5cm]{geometry}
\usepackage[utf8]{inputenc}
\usepackage{amsmath, amsfonts, amssymb} 

\begin{document}

\begin{enumerate}
    \item Considere a matriz 
    $
    \begin{bmatrix} %função que escreve a matriz com colchetes (b = brackets)
    1 & 10 & -5 \\ %cada coluna se diferencia com &, e cada mudança de linha se faz com \\
    6 & 7 & 8
    \end{bmatrix}
    $
    
    \item Considere a matriz 
    $
    \begin{pmatrix} %função que escreve a matriz com parênteses (p = parentesis)
    1 & 10 & -5 \\
    6 & 7 & 8 \\
    7 & 45 & 12
    \end{pmatrix}
    $
    
    \item Calcule
    $
    \begin{vmatrix} %função que escreve a matriz com barras verticais (v = vertical)
    1 & 10 & -5 \\
    6 & 7 & 8 \\
    7 & 45 & 12
    \end{vmatrix}
    $
    
    \item Considere a matriz 
    $M =
    \begin{bmatrix} 
    1 & 10 & -5 \\ 
    6 & 7 & 8 \\
    7 & 45 & 12
    \end{bmatrix}
    $.
    Calcule o que foi solicitado abaixo:
    
    \begin{enumerate}
        \item $\det M$ %função determinante
        \item $M^{-1}$
        \item $M^T$
    \end{enumerate}
    
     \item Considere a matriz $m \times n$ dada por 
    $
    \begin{bmatrix} 
        a_{11} & a_{12} & a_{13} & \cdots & a_{1n} \\ %colocar reticências horizontais
        a_{21} & a_{22} & a_{23} & \cdots & a_{2n} \\
        a_{31} & a_{32} & a_{33} & \cdots & a_{3n} \\
        \vdots & \vdots & \vdots & \ddots & \vdots \\%colocar reticências verticais \vdots e reticencias diagonais \ddots
        a_{m1} & a_{m2} & a_{m3} & \cdots & a_{mn}
    \end{bmatrix}
    $.
   
   \item Determine $x$, $y$ e $z$ na equação 
   $$
    \begin{bmatrix}
        1 & -2 & 4 \\ 
        5 & 2 & -2 \\
        6 & 1 & 8 
    \end{bmatrix}
    \begin{bmatrix}
        x \\ 
        y \\
        z 
    \end{bmatrix}
    =
    \begin{bmatrix}
        2 \\ 
        10 \\
        6 
    \end{bmatrix} 
   $$
   
    
\end{enumerate}

\end{document}