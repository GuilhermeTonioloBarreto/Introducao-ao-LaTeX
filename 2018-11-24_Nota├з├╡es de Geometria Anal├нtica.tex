%Notações de Geometria Analítica
\documentclass[a4paper, 12pt]{article}
\usepackage[top=2cm, bottom=2cm, left=2.5cm, right=2.5cm]{geometry}
\usepackage[utf8]{inputenc}
\usepackage{amsmath, amsfonts, amssymb} 

\begin{document}

\begin{enumerate}
    \item Seja o segmento $\overline{AB}$. %notação de segmento (traço em cima da(s) letra(s))
    A partir dele, podemos definir os segmentos orientados $\overrightarrow{AB}$ e $\overrightarrow{BA}$ %colocar segmentos orientados (setas em cima das letras). No caso, seta orientada a direita.
    Seja $\vec{AB}$ e $\vec{u}$. %coloca uma seta orientada a direita na palavra selecionada. Repare na compilação que essa função funciona melhor usando-se apenas uma letra (no caso, o vetor u), e não duas letras (como o vetor AB). Neste último caso, o melhor a ser usado é a função \overrightarrow{}.
    
    \item Sejam os vetores $\vec{u} = (1; \, -1 ;\, 2)$ e $\vec{v} = (2; \, 5 ;\, -4)$. Calcule o seguinte:
    
    \begin{enumerate}
        \item $\vec{u} \cdot \vec{v}$ %produto escalar
        \item $\langle \vec{u}, \, \vec{v}\rangle$ %produto escalar
        \item $\vec{u} \times \vec{v}$ %produto vetorial
        \item $|\vec{u}|$ %módulo do vetor
        \item $\|\vec{u}\|$ %módulo do vetor
        \item $\left\|\vec{U}\right\|$ %módulo do vetor (para letras maiúsculas)
        \item Verifique se $\vec{u} \perp \vec{v}$. %ver se um vetor é perpendicular ao outro
    \end{enumerate}
     
    \item Sejam os planos $\alpha : x - 2y + 6z - 3 = 0$ e $\beta : x - 2y + 6z - 3 = 0$ %notação da letra grega alpha e beta
    
    \item Sejam os vetores $\vec{u} = (x_0; \, y_0 ; \, z_0)$ e $\vec{v} = (x_1; \, y_1 ; \, z_1)$. Temos que:
    $$
    \vec{u} \times \vec{v}= 
    \begin{vmatrix}
        \vec{i} & \vec{j} & \vec{k} \\
        x_0 & y_0 & z_0 \\
        x_1 & y_1 & z_1
    \end{vmatrix}
    $$
\end{enumerate}

\end{document}