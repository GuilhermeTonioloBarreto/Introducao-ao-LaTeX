%Notações de Conjunto

\documentclass[a4paper, 12pt]{article}
\usepackage[top=2cm, bottom=2cm, left=2.5cm, right=2.5cm]{geometry}
\usepackage[utf8]{inputenc}
\usepackage{amsmath, amsfonts, amssymb} 

\begin{document}

\begin{enumerate}
    \item Sejam os conjuntos $A = \{1; 2; 3; 4\}$ %para escrever as chaves do conjunto, deve-se colocar \{ texto a ser escrito \}.
    
    \begin{enumerate}
        \item tipo de espacamento entre números: $A =\{1; 2; 3; 4 \}$ %um tipo de espacamento entre números numa equação
        \item tipo de espacamento entre números: $A =\{1;\, 2;\, 3;\, 4 \}$ %outro tipo de espaçamento entre números numa equação (colocando \, entre os números)
        \item tipo de espacamento entre números: $A =\{1;\; 2;\; 3;\; 4 \}$ %outro tipo de espaçamento entre números numa equação (colocando \; entre os números)
        \item tipo de espacamento entre números: $A =\{1;\ 2;\ 3;\ 4 \}$ %outro tipo de es paçamento entre números numa equação (colocando \ entre os números)
    \end{enumerate}
     
     \item Sejam os conjuntos $A = \{1; 2; 3; 4\}$, 
     $B = \{x \in \mathbb{Z} \;|\; -2 \leq x < 4 \}$ %B pertence ao universo dos inteiros
     e $\{C \in \mathbb{N} \,|\, x \geq 2 \}$. %C pertence ao universo dos naturais
     Responda aos itens abaixo:
     
     \begin{enumerate}
         \item $A \cap B$ %A intercecção B
         \item $B \cup C$ %B união C
         \item $A - C$ %A menos C
         \item $B \setminus C$ %B menos C
     \end{enumerate}
     
     \item Classifique em Verdadeiro ou Falso:
     
     \begin{enumerate}
         \item $\mathbb{Z} \subset \mathbb{N}$ %Z está contido em N
         \item $\mathbb{R} \supset \mathbb{Q}$ %R contém Q
         \item $\mathbb{Z} \not\subset \mathbb{N}$ %Z não está contido em N
         \item $\mathbb{R} \not\supset \mathbb{Q}$ %R não contém Q
         \item $0 \not\in \mathbb{I}$ %0 não pertence a I
         \item $0 \not\in \mathbb{R} \setminus \mathbb{Q}$ %0 não pertence a R - Q
         \item $\forall \ x \in \mathbb{N}$, temos $x \geq 0$ %para todo x pertencente aos números naturais, x é maior ou igual a 0
         \item $\exists \ x \in \mathbb{R}$ tal que $\sqrt{x} \not\in \mathbb{R}$ %Existe x pertencente aos Reais tal que a raiz quadrada de x não pertenca a R
         \item $7 \not\in \{x \in \mathbb{N} \,|\, x \textrm{ é par}\}$ %para se digitar um texto dentro de um ambiente matemático $$, deve-se usar a função \textrm{escreva seu texto aqui}
         \item $-5 \in \mathbb{R}^*_+$
         \item $0 \in \varnothing$ %0 pertence ao conjunto vazio
         
     \end{enumerate}
     
\end{enumerate}
\end{document}