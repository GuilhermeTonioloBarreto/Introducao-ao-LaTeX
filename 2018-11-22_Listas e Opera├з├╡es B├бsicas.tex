%Listas e Operações Básicas

\documentclass[a4paper, 12pt]{article}
\usepackage[top=2cm, bottom=2cm, left=2.5cm, right=2.5cm]{geometry}
\usepackage[utf8]{inputenc}
\usepackage{amsmath, amsfonts, amssymb} %ams = americam mathematic society. São alguns pacotes disponibilizados por essa sociedade

\begin{document}

\begin{enumerate} % criar uma lista numérica
    \item vem o texto
    
    \begin{enumerate} %criando listas numéricas dentro de outra lista
        \item aqui vem um texto
    
        \begin{enumerate}
            \item aqui vem um texto
            \item aqui vem mais outro texto
            \item aqui tem texto
        \end{enumerate}
        
        \item aqui vem mais outro texto
        \item aqui tem texto
    \end{enumerate}
    
    \item mais outro texto
    \item outro texto
\end{enumerate}

\begin{itemize} %criar listas não numeradas
        \item isso aqui é um item
        
        \begin{itemize} % criar listas não numeradas dentro de listas não numeradas
            \item sei lá
            \item não sei oque
            \item pode pá
        \end{itemize}
        
        \item isso aqui é outro item
        \item mais outro item
\end{itemize}

$a+b$ %soma

$a-b$ % subtração

$a \cdot b$ %multiplicação

$a \times b$ %multiplicação

$a \div b$ %divisão

fração expremida na linha: $\frac{a}{b}$ %fração: usando um $ apenas, a fração fica expremida, para que caiba na linha

uma linha de teste para ver como fica

fração não expremida na linha: $\dfrac{a}{b}$ %fração usando \dfrac{}{}. É uma função do pacote amsmath

$\sqrt{a}$ %colocar raiz quadrada

$\sqrt[3]{a}$ %colocar raiz enésima (no exemplo, raiz cúbica) de a 

$a^b$ %potência de um caractere

$a^{b+c}$ %potência com mais de um caractere

$a_1$, $a_2$, $a_3$ e $a_4$ %índice com um caractere

$a_{10}$ %índice com mais de um caractere
\end{document}